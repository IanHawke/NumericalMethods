% $Header: /cvsroot/latex-beamer/latex-beamer/solutions/conference-talks/conference-ornate-20min.en.tex,v 1.6 2004/10/07 20:53:08 tantau Exp $

\documentclass{beamer}
%\documentclass[handout]{beamer}
%\usepackage{pgfpages}
%\pgfpagesuselayout{2 on 1}[a4paper,border shrink=5mm]

% This file is a solution template for:

% - Talk at a conference/colloquium.
% - Talk length is about 20min.
% - Style is ornate.



% Copyright 2004 by Till Tantau <tantau@users.sourceforge.net>.
%
% In principle, this file can be redistributed and/or modified under
% the terms of the GNU Public License, version 2.
%
% However, this file is supposed to be a template to be modified
% for your own needs. For this reason, if you use this file as a
% template and not specifically distribute it as part of a another
% package/program, I grant the extra permission to freely copy and
% modify this file as you see fit and even to delete this copyright
% notice. 


\mode<presentation>
{
%  \usetheme{Warsaw}
%  \usetheme{Boadilla}
%  \usetheme{Goettingen}
%  \usetheme{Hannover}
%  \usetheme{Madrid}
%  \usetheme{Marburg}
%  \usetheme{Montpellier}
%  \usetheme{Pittsburgh}
  \usetheme{Hawke}
  % or ...

  \setbeamercovered{transparent}
  % or whatever (possibly just delete it)
}


\usepackage[english]{babel}
% or whatever

\usepackage[latin1]{inputenc}
% or whatever

\usepackage{times}
\usepackage[T1]{fontenc}
% Or whatever. Note that the encoding and the font should match. If T1
% does not look nice, try deleting the line with the fontenc.

\usepackage{multimedia}

%%%%%%
% My Commands
%%%%%%

\newcommand{\ml}{{\sc matlab}}
\newcommand{\bb}{{\boldsymbol{b}}}
\newcommand{\bx}{{\boldsymbol{x}}}
\newcommand{\bfm}[1]{{\boldsymbol{#1}}}

%%%%

\title[Lecture 3] % (optional, use only with long paper titles)
{Lecture 3 - Direct Methods and Gaussian Elimination} 

% \subtitle
% {Include Only If Paper Has a Subtitle}

\author[I. Hawke] % (optional, use only with lots of authors)
{I.~Hawke}
% - Give the names in the same order as the appear in the paper.
% - Use the \inst{?} command only if the authors have different
%   affiliation.

\institute[University of Southampton] % (optional, but mostly needed)
{
%  \inst{1}%
  School of Mathematics, \\
  University of Southampton, UK
}
% - Use the \inst command only if there are several affiliations.
% - Keep it simple, no one is interested in your street address.

\date[Semester 1] % (optional, should be abbreviation of conference name)
{MATH3018/6141, Semester 1}
% - Either use conference name or its abbreviation.
% - Not really informative to the audience, more for people (including
%   yourself) who are reading the slides online

\subject{Numerical methods}
% This is only inserted into the PDF information catalog. Can be left
% out. 



% If you have a file called "university-logo-filename.xxx", where xxx
% is a graphic format that can be processed by latex or pdflatex,
% resp., then you can add a logo as follows:

\pgfdeclareimage[height=0.5cm]{university-logo}{mathematics_7469}
\logo{\pgfuseimage{university-logo}}



% Delete this, if you do not want the table of contents to pop up at
% the beginning of each subsection:
%  \AtBeginSubsection[]
%  {
%    \begin{frame}<beamer>
%      \frametitle{Outline}
%      \tableofcontents[currentsection,currentsubsection]
%    \end{frame}
%  }
\AtBeginSection[]
{
  \begin{frame}<beamer>
    \frametitle{Outline}
    \tableofcontents[currentsection]
  \end{frame}
}


% If you wish to uncover everything in a step-wise fashion, uncomment
% the following command: 

%\beamerdefaultoverlayspecification{<+->}


\begin{document}

\begin{frame}
  \titlepage
\end{frame}

\section{Direct vs.\ Indirect}

\subsection{Direct vs.\ Indirect}

\begin{frame}
  \frametitle{Direct vs.\ Indirect}

  We want to solve the linear system
  \begin{equation*}
    A \bx = \bb
  \end{equation*}
  for (reasonably conditioned) arbitrary $A$. This can be done using a
  method that is either 
  \begin{itemize}
  \item Direct: a finite algorithm that produces the correct answer, or
  \item Indirect (or iterative): an infinite algorithm that steadily
    approaches the correct answer, maybe only asymptotically.
  \end{itemize}
  \pause

  \vspace{1ex}

  Direct is \emph{not} always best:
  \begin{enumerate}
  \item For a large matrix ($n \gtrsim 10^5$) a direct method may be
    impractical.
  \item Accumulation of round-off error may make the direct method
    considerably less accurate than the indirect method.
  \end{enumerate}

\end{frame}


\subsection{Gaussian Elimination}

\begin{frame}
  \frametitle{Gaussian Elimination}
  
  The classic direct method.  Find a simple problem to solve, and
  transform everything to that simple problem. Here make the matrix
  \emph{upper triangular}. \pause

  \begin{equation*}
    \begin{pmatrix}
      \color<5|handout:0>{red}1 & \color<5|handout:0>{red}2 & \color<5|handout:0>{red}3 \\
      0 & \color<4|handout:0>{red}4 & \color<4|handout:0>{red}5 \\
      0 & 0 & \color<3|handout:0>{red}6
    \end{pmatrix}
    \begin{pmatrix}
      \color<5|handout:0>{red}x_1 \\ \color<4|handout:0>{red}x_2 \\ \color<3|handout:0>{red}x_3
    \end{pmatrix}
    =
    \begin{pmatrix}
      \color<5|handout:0>{red}6 \\ \color<4|handout:0>{red}9 \\ \color<3|handout:0>{red}6
    \end{pmatrix}
  \end{equation*}
  is trivially solved by \emph{back-substitution}.\pause\ Solve from
  the bottom up:
  \begin{overlayarea}{\textwidth}{0.35\textheight}
    \only<3|handout:0>
    {
      \begin{align*}
        && {6 x_3} & {= 6} & {\Rightarrow x_3} & {= 1} \\
        && \phantom{4 x_2 + 5 x_3} & \phantom{= 9} && \\
        \phantom{\implies} && \phantom{4 x_2} & \phantom{= 4} & \phantom{\Rightarrow x_2} & \phantom{= 1} \\
        && \phantom{x_1 + 2 x_2 + 3 x_3} & \phantom{= 6} && \\
        \phantom{\implies} && \phantom{x_1} & \phantom{= 1.}
      \end{align*}
    }
    \only<4|handout:0>
    {
      \begin{align*}
        && { 6 x_3 } & { = 6 } & { \Rightarrow x_3 } & { = 1 } \\
        && { 4 x_2 + 5 x_3 } & { = 9 } && \\
        { \implies } && { 4 x_2 } & { = 4 } & { \Rightarrow x_2 } & {
          = 1 } \\
        && \phantom{ x_1 + 2 x_2 + 3 x_3 } & \phantom{ = 6 } && \\
        \phantom{ \implies } && \phantom{ x_1 } & \phantom{ = 1. }
      \end{align*}
    }
    \only<5>
    {
      \begin{align*}
        && {6 x_3} & {= 6} & {\Rightarrow x_3} & {= 1} \\
        && {4 x_2 + 5 x_3} & {= 9} && \\
        {\implies} && {4 x_2} & {= 4} & {\Rightarrow x_2} & {= 1} \\
        && {x_1 + 2 x_2 + 3 x_3} & {= 6} && \\
        {\implies} && {x_1} & {= 1.}
      \end{align*}
    }
  \end{overlayarea}

\end{frame}


\begin{frame}
  \frametitle{Converting to upper triangular form}
  
  Want upper triangular form. Matrix problem is 
  system of simultaneous equations. Eliminate appropriate variables
  from equations. \pause

  \vspace{2ex}

  Must consistently change right hand side. 
  Use \emph{augmented matrix}
  \begin{equation*}
    \left(
    \begin{array}{c|c}
      A & \bb
    \end{array}
    \right)
  \end{equation*}

  \vspace{1ex}

  Can e.g.\ scale equations (rows) or subtract one equation (row) from another.

  \vspace{2ex}

  Work from top to bottom, left to right, eliminating all entries
  below the diagonal.

\end{frame}

\begin{frame}
  \frametitle{Converting to upper triangular form - example}

  \begin{overlayarea}{\textwidth}{0.8\textheight}
    \only<1|handout:1>
    {
      Given a system form the augmented matrix:
      \begin{equation*}
        \begin{pmatrix}
          3 & 0 & 1 \\
          6 & 2 & 4 \\
          9 & 2 & 6
        \end{pmatrix}
        \bx =
        \begin{pmatrix}
          4 \\ 10 \\ 15
        \end{pmatrix} \rightarrow
        \left(
          \begin{array}{c c c|c}
            3 & 0 & 1 & 4 \\
            6 & 2 & 4 & 10 \\
            9 & 2 & 6 & 15      
          \end{array}
        \right).
      \end{equation*}
    }
    \only<2|handout:2>
    {
      \begin{equation*}
        \left(
          \begin{array}{c c c|c}
            3 & 0 & 1 & 4 \\
            6 & 2 & 4 & 10 \\
            9 & 2 & 6 & 15      
          \end{array}
        \right)
      \end{equation*}
      Take the first column. Eliminate entries below the diagonal
      using multiples of the first row:
      \begin{align*}
        \left(
          \begin{array}{c c c|c}
            3 & 0 & 1 & 4 \\
            6 & 2 & 4 & 10 \\
            9 & 2 & 6 & 15      
          \end{array}
        \right)
        & \rightarrow
        \left(
          \begin{array}{c c c|c}
            3 & 0 & 1 & 4 \\
            6 - \tfrac{6}{3} \times 3 & 2 - \tfrac{6}{3} \times 0 & 4
            - \tfrac{6}{3} \times 1 & 10 - \tfrac{6}{3} \times 4 \\
            9 - \tfrac{9}{3} \times 3& 2 - \tfrac{9}{3} \times 0 & 6 -
            \tfrac{9}{3} \times 1 & 15 - \tfrac{9}{3} \times 4
          \end{array}
        \right) \\
        & =     
        \left(
          \begin{array}{c c c|c}
            3 & 0 & 1 & 4 \\
            0 & 2 & 2 & 2 \\
            0 & 2 & 3 & 3
          \end{array}
        \right)
      \end{align*}
    }
    \only<3|handout:3>
    {
      \begin{equation*}
        \left(
          \begin{array}{c c c|c}
            3 & 0 & 1 & 4 \\
            6 & 2 & 4 & 10 \\
            9 & 2 & 6 & 15     
          \end{array}
        \right)
      \end{equation*}
      For the second column use multiples of the second row:
      \begin{align*}
        \left(
          \begin{array}{c c c|c}
            3 & 0 & 1 & 4 \\
            6 & 2 & 4 & 10 \\
            9 & 2 & 6 & 15      
          \end{array}
        \right)
        & \rightarrow
        \left(
          \begin{array}{c c c|c}
            3 & 0 & 1 & 4 \\
            0 & 2 & 2 & 2 \\
            0 & 2 & 3 & 3
          \end{array}
        \right) \\
        & \rightarrow
        \left(
          \begin{array}{c c c|c}
            3 & 0 & 1 & 4 \\
            0 & 2 & 2 & 2 \\
            0 & 2 - \tfrac{2}{2} \times 2 & 3 - \tfrac{2}{2} \times 2
            & 3 - \tfrac{2}{2} \times 2 
          \end{array}
        \right) \\
        & =
        \left(
          \begin{array}{c c c|c}
            3 & 0 & 1 & 4 \\
            0 & 2 & 2 & 2 \\
            0 & 0 & 1 & 1
          \end{array}
        \right) \Rightarrow \bx =
        \begin{pmatrix}
          1 \\ 0 \\ 1
        \end{pmatrix}.
      \end{align*}
    }
  \end{overlayarea}

\end{frame}


\subsection{Pivoting}

\begin{frame}
  \frametitle{Swapping rows and columns.}

  We can make our matrix simpler/better without changing the
  result. As the matrix defines simultaneous equations we can: \pause
  \begin{enumerate}
  \item scale any equation by a non-zero scalar number.  Equivalent to
    multiplying a row in the \emph{augmented} matrix by a
    number \pause
  \item swap the order of the equations. Equivalent to
    swapping rows in the \emph{augmented} matrix, \pause
  \item swap the order of the unknown variables.  Equivalent to
    swapping columns in the matrix. \pause
  \end{enumerate}
  \begin{overlayarea}{\textwidth}{0.35\textheight}
    \only<5-7> 
    { 
      Swapping rows is called \emph{pivoting}. Why do it? Trivial
      reason: Gaussian elimination requires dividing by a matrix
      coefficient, which must be non-zero.
    } \only<6|handout:1> {
      \begin{equation*}
        A = \left(
          \begin{array}{c c c |c}
            1 & 1 & 1 & 1 \\
            0 & 0 & 2 & 1 \\
            0 & 1 & 1 & 2
          \end{array} \right).
      \end{equation*}
      Without pivoting we would need to compute $1/0$: disaster!
    }
    \only<7|handout:2>
    {
      \begin{equation*}
        A = \left(
          \begin{array}{c c c |c}
            1 & 1 & 1 & 1 \\
            0 & 0 & 2 & 1 \\
            0 & 1 & 1 & 2
          \end{array} \right) \rightarrow
        \left(
          \begin{array}{c c c |c}
            1 & 1 & 1 & 1 \\
            0 & 1 & 1 & 2 \\
            0 & 0 & 2 & 1
          \end{array} \right) 
      \end{equation*}
      The row swap avoids the zero division.
    }      
  \end{overlayarea}

\end{frame}

\begin{frame}
  \frametitle{Pivoting for accuracy}
  
  A more subtle error appears can appear (here $\varepsilon \ll 1$).
  {\footnotesize
  \begin{equation*}
    \left(
      \begin{array}{c c|c}
        \varepsilon & 1 & 1 \\
        1 & 1 & 2
      \end{array}
      \right) \rightarrow     
      \left(
      \begin{array}{c c|c}
        \varepsilon & 1 & 1 \\
        0 & 1 - \varepsilon^{-1} & 2 - \varepsilon^{-1}
      \end{array}
    \right) \simeq
    \left(
      \begin{array}{c c|c}
        \varepsilon & 1 & 1 \\
        0 & - \varepsilon^{-1} & - \varepsilon^{-1}
      \end{array}
    \right) \Rightarrow \bx  =
    \begin{pmatrix}
      0 \\ 1
    \end{pmatrix}.
  \end{equation*}} \pause
  The correct answer is $\bx = (1, 1)^T$. The  system has good
  condition number, but division by small $\varepsilon$ destroys
  precision. \pause
  
  \vspace{1ex}

  Instead \emph{pivot} by swapping rows: we have the system
  \begin{equation*}
    \left(
      \begin{array}{c c|c}
        1 & 1 & 2 \\
        \varepsilon & 1 & 1 
      \end{array}
    \right)  \Rightarrow \bx \simeq
    \begin{pmatrix}
      1 \\ 1
    \end{pmatrix}.
  \end{equation*}
  The pivot should be as large in magnitude as possible.

\note{

  Do the calculation explicitly:
  \begin{align*}
    &&
    \left(
      \begin{array}{c c|c}
        \varepsilon & 1 & 1 \\
        1 & 1 & 2        
      \end{array}
    \right) & \rightarrow
    \left(
      \begin{array}{c c|c}
        \varepsilon & 1 & 1 \\
        0 & 1 - \varepsilon^{-1} & 2 - \varepsilon^{-1}
      \end{array}
    \right) \\
    &&& \simeq
    \left(
      \begin{array}{c c|c}
        \varepsilon & 1 & 1 \\
        0 & - \varepsilon^{-1} & - \varepsilon^{-1}
      \end{array}
    \right) \\
    \Rightarrow&& \bx & =
    \begin{pmatrix}
      0 \\ 1
    \end{pmatrix}.
  \end{align*}
  However, this is an artefact; the pivoted calculation gives
  \begin{align*}
    &&
    \left(
      \begin{array}{c c|c}
        1 & 1 & 2 \\
        \varepsilon & 1 & 1
      \end{array}
    \right) & \rightarrow
    \left(
      \begin{array}{c c|c}
        1 & 1 & 2 \\
        0 & 1 - \varepsilon & 1 - 2 \varepsilon
      \end{array}
    \right) \\
    &&& \simeq
    \left(
      \begin{array}{c c|c}
        1 & 1 & 2 \\
        0 & 1 & 1
      \end{array}
    \right) \\
    \Rightarrow&& \bx & =
    \begin{pmatrix}
      1 \\ 1
    \end{pmatrix},
  \end{align*}
  which is correct in the limit $\varepsilon \rightarrow 0$.
}

\end{frame}

\begin{frame}
  \frametitle{Partial vs. total pivoting}
    
  {
    \footnotesize
    \renewcommand{\arraystretch}{-1.2}
  \begin{equation*}
    \left(
      \begin{array}{c c c c c c|c}
        1 & \cdots & \cdots & \cdots & \cdots & \cdots & \vdots \\
        0 & \ddots & \cdots & \cdots & \cdots & \cdots & \vdots \\
        0 & 0 & 2 & \cdots & \cdots & \cdots & \vdots \\
        0 & 0 & \color<1|handout:0>{red}-10 & \cdots & \cdots & \cdots & \vdots \\
        0 & 0 & \vdots & \cdots & \cdots & \color<2-3|handout:0>{red}100 & \vdots \\
        0 & 0 & \vdots & \cdots & \cdots & \cdots & \vdots \\
      \end{array}
    \right)
  \end{equation*}
  }

  \emph{Partial pivoting}: For row $i$ find row ($j \geq i$) with
  largest coefficient $a_{ji}$ (in absolute value). Swap row $j$ with
  row $i$. \pause

  \vspace{1ex}

  \emph{Total pivoting}: For row $i$ find row ($j \geq i$) with
  largest coefficient $a_{jk}$ (in absolute value). Swap row $j$ with
  row $i$ \emph{and} column $i$ with column $k$. \pause

  \vspace{1ex}

  Total pivoting is complex, expensive and usually
  unnecessary. Partial pivoting is often essential.

\end{frame}

\begin{frame}
  \frametitle{Example}
  
  The augmented matrix is
  \begin{equation*}
    A = \left(
      \begin{array}{c c c|c}
        1 & 2 & 2 & 5 \\
        2 & 1 & 0 & 3 \\
        3 & -1 & 1 & 3 
      \end{array}
      \right).
  \end{equation*}\pause
  \begin{overlayarea}{\textwidth}{0.7\textheight}
    \only<2|handout:1>
    {
      Remove entries below the diagonal in the first column.  Pivot to
      get biggest coefficient on diagonal. Swapping rows:
      \begin{equation*}
        \left(
          \begin{array}{c c c|c}
            1 & 2 & 2 & 5 \\
            2 & 1 & 0 & 3 \\
            \color{red}3 & -1 & 1 & 3
          \end{array}
        \right) \xrightarrow{\text{Row 1} \leftrightarrow \text{Row 3}}
        \left(
          \begin{array}{c c c|c}
            3 & -1 & 1 & 3 \\
            2 & 1 & 0 & 3 \\
            1 & 2 & 2 & 5 
          \end{array}
        \right).
      \end{equation*}
    }
    \only<3|handout:2>
    {
      Eliminate entries in first column:
      {
        \renewcommand{\arraystretch}{1.2}
      \begin{equation*}
        \left(
          \begin{array}{c c c|c}
            3 & -1 & 1 & 3 \\
            2 & 1 & 0 & 3 \\
            1 & 2 & 2 & 5 
          \end{array} \right) \xrightarrow[\substack{\text{Row 2}
          \rightarrow \text{Row 2} - \tfrac{2}{3}\text{Row 1,} \\
          \text{Row 3} \rightarrow \text{Row 3} -
          \tfrac{1}{3}\text{Row 1}}]{\text{eliminate column 1}}
        \left(
          \begin{array}{c c c|c}
            3 & -1 & 1 & 3 \\
            0 & \tfrac{5}{3} & -\tfrac{2}{3} & 1 \\
            0 & \tfrac{7}{3} & \tfrac{5}{3} & 4 
          \end{array}\right) .
      \end{equation*}
    }
    }
    \only<4|handout:3>
    {
      To simplify algebra rescale the last two rows: 
      {
      \renewcommand{\arraystretch}{1.2}
      \begin{equation*}
        \left(
          \begin{array}{c c c|c}
            3 & -1 & 1 & 3 \\
            0 & \tfrac{5}{3} & -\tfrac{2}{3} & 1 \\
            0 & \tfrac{7}{3} & \tfrac{5}{3} & 4 
          \end{array} \right) \xrightarrow{\text{multiply rows 2, 3 by
          3}}  \left(
          \begin{array}{c c c|c}
            3 & -1 & 1 & 3 \\
            0 & 5 & -2 & 3 \\
            0 & 7 & 5 & 12 
          \end{array} \right) .
      \end{equation*}
    }
    }
    \only<5|handout:4>
    {
      Done first column. For the second column, \emph{consider only
        rows 2 and 3}. Once again, pivot:
      \begin{equation*}
        \left(
          \begin{array}{c c c|c}
            3 & -1 & 1 & 3 \\
            0 & 5 & -2 & 3 \\
            0 & \color{red}7 & 5 & 12 
          \end{array} \right) \xrightarrow{\text{Row 2}
          \leftrightarrow \text{Row 3}}  \left(
          \begin{array}{c c c|c}
            3 & -1 & 1 & 3 \\
            0 & 7 & 5 & 12 \\
            0 & 5 & -2 & 3
          \end{array} \right) .
      \end{equation*}
    }
    \only<6|handout:5>
    {
      Finally eliminate the second column:
      \begin{align*}
        && \left(
          \begin{array}{c c c|c}
            3 & -1 & 1 & 3 \\
            0 & 7 & 5 & 12 \\
            0 & 5 & -2 & 3
          \end{array} \right) & \xrightarrow[\text{Row 3} \rightarrow
        \text{Row 3} - \tfrac{5}{7} \text{Row 2}]{\text{eliminate
            column 2}}  \left(
          \begin{array}{c c c|c}
            3 & -1 & 1 & 3 \\
            0 & 7 & 5 & 12 \\
            0 & 0 & -\tfrac{39}{7} & -\tfrac{39}{7}
          \end{array} \right) \\ 
        && \Rightarrow
        \bx & =
        \begin{pmatrix}
          1 \\ 1 \\ 1
        \end{pmatrix}.
      \end{align*}
    }
  \end{overlayarea}

\end{frame}

\subsection{Operation count}

\begin{frame}
  \frametitle{Operation count}

  How expensive is Gaussian elimination?

  \vspace{1ex}

  Only count multiplication/division. Consider a large $n \times n$
  matrix and approximate.

  \vspace{1ex}

  For row $k$ we eliminate $n-k$ entries by multiplying the pivot by
  the row, hence $n(n-k)$ operations. Total is
  % 
  \begin{equation*}
    \sum_{k=1}^{n-1} n (n - k) = \tfrac{1}{2} n^2 (n - 1).
  \end{equation*}
  % 
  For large $n$ this is $\sim \tfrac{1}{2} n^3$.
  
  \vspace{1ex}

  Back substitution is $\sim n^2$ so small in comparison.

\end{frame}

\section{Summary}

\subsection{Summary}

 \begin{frame}
   \frametitle{Summary}

   \begin{itemize}
   \item Direct methods solve linear systems in a finite number of
     steps; indirect or iterative methods may be infinite but can be
     truncated to give an approximate solution.
   \item Indirect methods are often faster to reach a given accuracy;
     accumulated round-off error may mean direct methods are not
     ``exact''.
   \item Gaussian elimination is a direct method that is an $n^3$
     algorithm, where $n$ is the matrix size.
   \item Pivoting increases accuracy (reduces error due to accumulated
     round-off). \emph{It does not help if the problem is
       ill-conditioned}.
   \item Partial pivoting is preferred to total pivoting in practice.
   \item All operations (except swapping columns) are applied to the
     \emph{augmented} matrix.
   \end{itemize}
   
 \end{frame}



\end{document}



%%% Local Variables: 
%%% mode: latex
%%% TeX-master: t
%%% End: 
