\documentclass{beamer}


\mode<presentation>
{
  \usetheme{Hawke}
  \setbeamercovered{transparent}
}

\usepackage[english]{babel}
\usepackage[latin1]{inputenc}
\usepackage{times}
\usepackage[T1]{fontenc}
\usepackage{multimedia}


%%%%%%
% My Commands
%%%%%%

\newcommand{\bb}{{\boldsymbol{b}}}
\newcommand{\bx}{{\boldsymbol{x}}}
\newcommand{\by}{{\boldsymbol{y}}}
\newcommand{\bfm}[1]{{\boldsymbol{#1}}}
\newcommand{\pda}[2]{\frac{\partial{#1}}{\partial{#2}}}
\newcommand{\pdb}[2]{\frac{\partial^2{#1}}{\partial{#2}^2}}
\newcommand{\pdc}[3]{\frac{\partial^2{#1}}{\partial{#2}\partial{#3}}}


%%%%

\title[Lecture 28] % (optional, use only with long paper titles)
{Lecture 28 - Von Neumann stability 2}

\author[I. Hawke] % (optional, use only with lots of authors)
{I.~Hawke}
\institute[University of Southampton] % (optional, but mostly needed)
{
  School of Mathematics, \\
  University of Southampton, UK
}

\date[Semester 1] % (optional, should be abbreviation of conference name)
{MATH3018/6141, Semester 1}

\subject{Numerical methods}

\pgfdeclareimage[height=0.5cm]{university-logo}{mathematics_7469}
\logo{\pgfuseimage{university-logo}}

\AtBeginSection[]
{
  \begin{frame}<beamer>
    \frametitle{Outline}
    \tableofcontents[currentsection]
  \end{frame}
}


\begin{document}

\begin{frame}
  \titlepage
\end{frame}

\section{Stability}

\subsection{Evidence}

\begin{frame}
  \frametitle{Partial differential equations}

  Partial differential equations (PDEs) involve derivatives of
  functions of more than one variable, say $u(x, y)$ or $y(t,
  x)$. Hence more complex behaviour and more interesting
  physics. \pause

  \vspace{1ex}

  Only look at linear problems.  Also only consider finite difference
  methods: simple to analyse but not always competitive. \pause

  \vspace{1ex}

  Previously looked at simple finite difference algorithms applied to
  heat and advection equations. At present have no explanation for
  which worked and which did not; for that we need stability theory.

\end{frame}

\begin{frame}
  \frametitle{The results}

  \begin{center}
    \begin{tabular}{c|c c}
      Algorithm & Heat equation & Advection equation \\ \hline
      FTCS & Stable $s < 1/2$ & Unstable \\
      BTCS & Stable $\forall s$ & ?? \\
      FTBS & ?? & Stable $c < 1$
    \end{tabular}
  \end{center}

  Here $s = k \delta / h^2$ for the heat equation
  \begin{equation*}
    \partial_t y - k \partial_{x x} y = 0,
  \end{equation*}
  and $c = v \delta / h$ for the advection equation
  \begin{equation*}
    \partial_t y + v \partial_x y = 0.
  \end{equation*}

\end{frame}


\section{Von Neumann stability theory}

\subsection{Von Neumann stability theory}

\begin{frame}
  \frametitle{Determining stability}

  Von Neumann stability method for linear algorithms and equations:
  \begin{enumerate}
  \item<2-> Make ansatz that solutions of the finite difference
    equation have form
    \begin{equation*}
      y_{\ell}^k =  e^{\alpha j \ell h} q^k, \qquad (\, j = \sqrt{-1}, \alpha \in {\mathbb R}, q \in {\mathbb C} ).
    \end{equation*}
  \item<3-> Substitute this solution into the finite difference
    equation.
  \item<4-> Determine $q$ as a function of $\alpha$, the grid spacings
    $h,\delta$, and known numbers (such as the advection velocity
    $v$).
  \item<5-> Determine under what conditions we have stability, i.e.\ when
    \begin{equation*}
      |q| < 1 \quad \forall \alpha.
    \end{equation*}
  \end{enumerate}

\end{frame}


\section{Von Neumann stability in practise}


\subsection{Von Neumann stability in practise}

\begin{frame}
  \frametitle{BTCS for the heat equation}

  We look at BTCS in the form
  \begin{equation*}
    (1 + 2 s) y_i^{n+1} = y_i^n + s \left( y_{i+1}^{n+1} +
      y_{i-1}^{n+1} \right).
  \end{equation*} \pause
  Use ansatz $y_{\ell}^k =  \exp (\alpha j \ell h) q^k$
  and removing common factors:
  \begin{equation*}
    (1 + 2 s) q = 1 + s q \left( e^{\alpha j h} +  e^{-\alpha j h}
    \right).
  \end{equation*} \pause
  Rewrite this as
  \begin{align*}
    && q & = 1 + 2 s \left( e^{\alpha j h} +  e^{-\alpha j h} - 2 \right)
    q \\
    && & = 1 - 4 s q \sin^2 \left( \tfrac{\alpha h}{2} \right) \\
    & \implies & q & = \left( 1 + 4 s \sin^2 \left( \tfrac{\alpha
          h}{2} \right) \right)^{-1},
  \end{align*} \pause
  which is \emph{always} less than 1. Hence BTCS is unconditionally
  stable.

\end{frame}

\begin{frame}
  \frametitle{FTCS for the advection equation}

  Ignore difference in \emph{differential} equation; only need
  analysis at level of finite difference equation:
  \begin{equation*}
    y_i^{n+1} = y_i^n - \frac{c}{2} \left( y_{i+1}^n - y_{i-1}^n
    \right).
  \end{equation*}
  Use ansatz $y_{\ell}^k = \exp (\alpha j \ell h) q^k$ and remove
  common factors:
  \begin{align*}
    && q & = 1 - \frac{c}{2} \left( e^{\alpha j h} - e^{-\alpha j h}
    \right) \\
    & \implies & q & = 1 - j c \sin ( \alpha h ).
  \end{align*} \pause
  As this is complex the modulus is
  \begin{equation*}
    |q|^2 = 1 + c^2 \sin^2 ( \alpha h )
  \end{equation*}
  which is always greater than 1: method is unconditionally unstable.

\end{frame}

\begin{frame}
  \frametitle{FTBS for the advection equation}

  Finally for the upwind method, FTBS, we had the difference equation
  \begin{equation*}
    y_i^{n+1} = y_i^n - c \left( y_{i}^n - y_{i-1}^n \right).
  \end{equation*}
  Use ansatz $y_{\ell}^k =  \exp (\alpha j \ell h) q^k$
  and remove common factors:
  \begin{equation*}
    q = 1 - c \left( 1 - e^{-\alpha j h} \right).
  \end{equation*} \pause

  Describes circle radius $c$ in the complex plane, centre at $1 -
  c$. \pause If $c > 1$ there are finite values of $\alpha$ for which
  $|q| > 1$, leading to instability; if $c < 1$ then $q$ is at most 1,
  leading to stability.

\end{frame}

\section{Summary}

\subsection{Summary}

\begin{frame}
  \frametitle{Summary}

  \begin{itemize}
  \item Von Neumann stability analysis is the key tool for
    understanding the results seen ``experimentally'' in the last
    lecture.
  \item By looking at the growth or decay of individual frequency
    modes we can check the stability of the method; Lax's theorem then
    proves convergence.
  \item For complex equations and / or algorithms, it is often very
    difficult or impossible to apply even Von Neumann stability
    analysis.
  \item If an algorithm does not work in the linear case then it is
    highly unlikely to work for more complex problems!
  \end{itemize}

\end{frame}

\end{document}
