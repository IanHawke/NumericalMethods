% $Header: /cvsroot/latex-beamer/latex-beamer/solutions/conference-talks/conference-ornate-20min.en.tex,v 1.6 2004/10/07 20:53:08 tantau Exp $

\documentclass{beamer}
%\documentclass[handout]{beamer}
%\usepackage{pgfpages}
%\pgfpagesuselayout{2 on 1}[a4paper,border shrink=5mm]

% This file is a solution template for:

% - Talk at a conference/colloquium.
% - Talk length is about 20min.
% - Style is ornate.



% Copyright 2004 by Till Tantau <tantau@users.sourceforge.net>.
%
% In principle, this file can be redistributed and/or modified under
% the terms of the GNU Public License, version 2.
%
% However, this file is supposed to be a template to be modified
% for your own needs. For this reason, if you use this file as a
% template and not specifically distribute it as part of a another
% package/program, I grant the extra permission to freely copy and
% modify this file as you see fit and even to delete this copyright
% notice.


\mode<presentation>
{
%  \usetheme{Warsaw}
%  \usetheme{Boadilla}
%  \usetheme{Goettingen}
%  \usetheme{Hannover}
%  \usetheme{Madrid}
%  \usetheme{Marburg}
%  \usetheme{Montpellier}
%  \usetheme{Pittsburgh}
  \usetheme{Hawke}
  % or ...

  \setbeamercovered{transparent}
  % or whatever (possibly just delete it)
}


\usepackage[english]{babel}
% or whatever

\usepackage[latin1]{inputenc}
% or whatever

\usepackage{times}
\usepackage[T1]{fontenc}
% Or whatever. Note that the encoding and the font should match. If T1
% does not look nice, try deleting the line with the fontenc.

\usepackage{multimedia}


%%%%%%
% My Commands
%%%%%%

\newcommand{\ml}{{\sc matlab}}
\newcommand{\bb}{{\boldsymbol{b}}}
\newcommand{\bx}{{\boldsymbol{x}}}
\newcommand{\by}{{\boldsymbol{y}}}
\newcommand{\bfm}[1]{{\boldsymbol{#1}}}
\newcommand{\oda}[2]{\frac{\text{d}{#1}}{\text{d}{#2}}}
\newcommand{\pda}[2]{\frac{\partial{#1}}{\partial{#2}}}
\newcommand{\pdb}[2]{\frac{\partial^2{#1}}{\partial{#2}^2}}
\newcommand{\pdc}[3]{\frac{\partial^2{#1}}{\partial{#2}\partial{#3}}}


%%%%

\title[Lecture 25] % (optional, use only with long paper titles)
{Lecture 25 - Upwind Methods}

% \subtitle
% {Include Only If Paper Has a Subtitle}

\author[I. Hawke] % (optional, use only with lots of authors)
{I.~Hawke}
% - Give the names in the same order as the appear in the paper.
% - Use the \inst{?} command only if the authors have different
%   affiliation.

\institute[University of Southampton] % (optional, but mostly needed)
{
%  \inst{1}%
  School of Mathematics, \\
  University of Southampton, UK
}
% - Use the \inst command only if there are several affiliations.
% - Keep it simple, no one is interested in your street address.

\date[Semester 1] % (optional, should be abbreviation of conference name)
{MATH3018/6141, Semester 1}
% - Either use conference name or its abbreviation.
% - Not really informative to the audience, more for people (including
%   yourself) who are reading the slides online

\subject{Numerical methods}
% This is only inserted into the PDF information catalog. Can be left
% out.



% If you have a file called "university-logo-filename.xxx", where xxx
% is a graphic format that can be processed by latex or pdflatex,
% resp., then you can add a logo as follows:

\pgfdeclareimage[height=0.5cm]{university-logo}{mathematics_7469}
\logo{\pgfuseimage{university-logo}}



% Delete this, if you do not want the table of contents to pop up at
% the beginning of each subsection:
%  \AtBeginSubsection[]
%  {
%    \begin{frame}<beamer>
%      \frametitle{Outline}
%      \tableofcontents[currentsection,currentsubsection]
%    \end{frame}
%  }
\AtBeginSection[]
{
  \begin{frame}<beamer>
    \frametitle{Outline}
    \tableofcontents[currentsection]
  \end{frame}
}


% If you wish to uncover everything in a step-wise fashion, uncomment
% the following command:

%\beamerdefaultoverlayspecification{<+->}


\begin{document}

\begin{frame}
  \titlepage
\end{frame}

\section{Hyperbolic equations and information}

\subsection{Information propagation}

\begin{frame}
  \frametitle{Partial differential equations}

  Partial differential equations (PDEs) involve derivatives of
  functions of more than one variable, say $u(x, y)$ or $y(t,
  x)$. Hence more complex behaviour and more interesting
  physics. \pause

  \vspace{1ex}

  Only look at linear problems.  Also only consider finite difference
  methods: simple to analyse but not always competitive. \pause

  \vspace{1ex}

  Only stable and convergent method for hyperbolic equations seen is
  Forward Time, Backwards Space (FTBS), sometimes called the
  \emph{upwind} method. Von Neumann stability ``explains'' failure of
  the FTCS method, but does not help in producing more accurate
  algorithms.

\end{frame}

\begin{frame}
  \frametitle{Characteristics for the advection equation}
  \begin{overlayarea}{\textwidth}{0.9\textheight}
    \only<1|handout:0>
    {
      Focus on advection equation
      \begin{equation*}
        \partial_t y + v \partial_x y = 0.
      \end{equation*}
      As hyperbolic,
      write in terms of convection or advection number $c = v
      \delta / h$  ($\delta$  time step, $h$ space step).
    }
    \only<2->
    {
      \begin{equation*}
        \partial_t y + v \partial_x y = 0.
      \end{equation*}
      Hyperbolic equations propagate
      information.   There are curves,
      $x(t)$, along which $y$ \emph{does not change}.
    }
    \only<3->
    {
      Look at $y\left( x(t), t \right)$.
    }
    \only<4->
    {
      If the solution does not change then
      \begin{align*}
         0 & = \oda{}{t} y\left( x(t), t \right) \\
           & = \oda{x(t)}{t} \partial_x y + \partial_t y \\
           & = \left( \oda{x(t)}{t} - v \right) \partial_x y. \\
        \Rightarrow \quad x(t) & = v t.
      \end{align*}
    }
    \only<5->
    {
      Information is propagated \emph{without
        changing} along straight lines (\emph{characteristic curves})
      in $(x, t)$ with slope $v$.
    }
  \end{overlayarea}

\end{frame}


\subsection{Upwind methods}


\begin{frame}
  \frametitle{Reinterpreting FTBS}

  \begin{center}
    \includegraphics<1|handout:0>[width=0.9\textwidth]{figures/Grid1a}
    \includegraphics<2|handout:0>[width=0.9\textwidth]{figures/Grid2}
    \includegraphics<3-4|handout:1>[width=0.9\textwidth]{figures/Grid3}
    \includegraphics<5|handout:2>[width=0.9\textwidth]{figures/Grid3a}
  \end{center}
  \begin{overlayarea}{\textwidth}{0.35\textheight}
    \only<1-3|handout:1>
    {
      Look at characteristics for advection equation on a grid.
    }
    \only<2-3|handout:1>
    {
      Given value of $y^{n+1}_i$, could follow characteristic
      \emph{back in time} to known data at $t^n$.
    }
    \only<3|handout:1>
    {
      Will not land on a grid point, so must interpolate.
    }
    \only<4-|handout:2>
    {
      Linear interpolation gives
      \begin{equation*}
        y_i^{n+1} = (1 - \frac{v \delta}{h}) y_i^n + \frac{v \delta}{h}
        y_{i-1}^n.
      \end{equation*}
    }
    \only<5|handout:2>
    {
      This is exactly FTBS.
    }
  \end{overlayarea}
\end{frame}

\begin{frame}
  \frametitle{Interpreting the stability result}

  \begin{center}
    \includegraphics<1|handout:1>[width=0.9\textwidth]{figures/Grid4}
    \includegraphics<2|handout:2>[width=0.9\textwidth]{figures/Grid4a}
    \includegraphics<3|handout:3>[width=0.9\textwidth]{figures/Grid5a}
    \includegraphics<4|handout:4>[width=0.9\textwidth]{figures/Grid5b}
  \end{center}
  \begin{overlayarea}{\textwidth}{0.35\textheight}
    \only<1-2|handout:1-2>
    {
      If timestep $\delta$ too large ($c > 1$) then characteristic,
      traced back to known data, does not fall between $x_i$ and
      $x_{i-1}$.
    }
    \only<2|handout:2>
    {
      Would be \emph{extrapolating} using FTBS; generically
      unstable. Known as the \emph{Courant-Friedrichs-Lewy} (CFL)
      condition.
    }
    \only<3-|handout:3->
    {
      If $v<0$ then FTBS is an extrapolation; Von Neumann analysis
      shows FTBS unconditionally unstable (note $c < 0$).
    }
    \only<4|handout:4>
    {
      Should be clear that FTFS will work in this case.
    }
  \end{overlayarea}

\end{frame}


\begin{frame}
  \frametitle{Upwind methods}

  The direction of information propagation depends on sign of $v$;
  called \emph{wind direction} (climate analogy). \pause
  Information propagates with wind, or \emph{upwind}:
  stable numerical algorithms take this into account. \pause

  \vspace{1ex}

  Simplest upwind algorithm: choose FTBS or FTFS depending on sign of
  $v$.
  \begin{align*}
    v & > 0 \quad \Rightarrow  \text{FTBS} \\
    v & < 0 \quad \Rightarrow  \text{FTFS} .
  \end{align*} \pause

  If advection velocity depends on \emph{space}, $v \equiv v(x)$, make
  choice \emph{pointwise}. \pause For each $x_i$ check sign of
  $v(x_i)$. Then update $y_i^{n+1}$ depending on sign of $v(x_i)$.

\end{frame}


\section{Nonlinear equations}


\subsection{Burger's equation}

\begin{frame}
  \frametitle{Burger's equation}

  Nonlinear hyperbolic equations can produce complex behaviour from
  simple initial data. \pause Toy problem: \emph{Burger's equation}
  \begin{equation*}
    \partial_t y + \partial_x \left( \tfrac{1}{2} y^2 \right) = 0.
  \end{equation*} \pause
  A linearization of Euler equations, useful in gasdynamics. \pause

  \vspace{1ex}

  If $y$ remains differentiable, can be written as
  \begin{equation*}
    \partial_t y + y \partial_x y = 0.
  \end{equation*} \pause
  Looks like advection equation, except advection velocity $v$ varies
  in space -- it is the solution $y$ itself!

\end{frame}

\begin{frame}
  \frametitle{Evolving to form shocks}

  \begin{columns}
    \begin{column}{0.4\textwidth}
      Start with smooth initial data. \pause We ``trace along the
      characteristics'': the \emph{form} of the data changes. \pause
      It steadily steepens. \pause

      \vspace{1ex}

      Eventually function not well-defined: in reality, a \emph{shock}
      forms.
    \end{column}
    \begin{column}{0.6\textwidth}
      \begin{center}
        \includegraphics<1|handout:0>[width=\textwidth]{figures/BurgersChar0}
        \includegraphics<2|handout:1>[width=\textwidth]{figures/BurgersChar1}
        \includegraphics<3|handout:0>[width=\textwidth]{figures/BurgersChar2}
        \includegraphics<4-|handout:2>[width=\textwidth]{figures/BurgersChar3}
      \end{center}
    \end{column}
  \end{columns}

\end{frame}

\begin{frame}
  \frametitle{Numerical evolutions of Burger's equation}

  \begin{columns}
    \begin{column}{0.4\textwidth}
      Evolve using upwind methods. If $y_i$ is positive use FTBS,
      otherwise FTFS. \pause

      \vspace{1ex}

      At low resolution see profile steepen \pause[5] and
      discontinuity form.  \pause Increase resolution \pause and see
      similar behaviour, better resolved \pause[10] leading to the
      shock.
    \end{column}
    \begin{column}{0.6\textwidth}
      \begin{center}
        \includegraphics<1|handout:0>[width=\textwidth]{figures/UpwindBurger1_0.png}
        \includegraphics<2|handout:0>[width=\textwidth]{figures/UpwindBurger1_2.png}
        \includegraphics<3|handout:0>[width=\textwidth]{figures/UpwindBurger1_4.png}
        \includegraphics<4|handout:0>[width=\textwidth]{figures/UpwindBurger1_8.png}
        \includegraphics<5|handout:1>[width=\textwidth]{figures/UpwindBurger1_16.png}
        \includegraphics<6|handout:0>[width=\textwidth]{figures/UpwindBurger5_0.png}
        \includegraphics<7|handout:0>[width=\textwidth]{figures/UpwindBurger5_32.png}
        \includegraphics<8|handout:0>[width=\textwidth]{figures/UpwindBurger5_64.png}
        \includegraphics<9|handout:0>[width=\textwidth]{figures/UpwindBurger5_128.png}
        \includegraphics<10|handout:2>[width=\textwidth]{figures/UpwindBurger5_256.png}
      \end{center}
    \end{column}
  \end{columns}

\end{frame}

\section{Summary}

\subsection{Summary}

\begin{frame}
  \frametitle{Summary}

  \begin{itemize}
  \item Upwind methods take into account the direction and speed of
    information propagation.
  \item As they rely on information propagation they only work for
    hyperbolic problems.
  \item Most upwind methods are explicit and conditionally stable,
    requiring the CFL condition.
  \item The upwind condition allows for the reinterpretation of, e.g.,
    FTBS.
  \item Upwind methods can be extended to problems where the advection
    velocity varies in space, or to genuinely nonlinear problems.
  \item Genuinely nonlinear problems that involve discontinuities
    forming are typically solved using upwind methods; extensions to
    higher order accuracy are non-trivial work.
  \end{itemize}

\end{frame}

\end{document}



%%% Local Variables:
%%% mode: latex
%%% TeX-master: t
%%% End:
